\documentclass[10pt,a4paper]{article}
\usepackage[latin1]{inputenc}
\usepackage[spanish]{babel}
\usepackage{amsmath}
\usepackage{amsfonts}
\usepackage{amssymb}
\usepackage{graphicx}
\usepackage[left=2cm,right=2cm,top=2cm,bottom=2cm]{geometry}
\author{Leonel Castaneda, Castaneda161618@unis.edu.gt}
\title{Hoja de trabajo No.1}
\date{23 de enero del 2018}



\begin{document}
\title{Hoja de trabajo No.1}
\maketitle{}


\section{"Que hacer"}
\begin{itemize}

\item string Fecha (Cuando se va a realizar la actividad).
\item int Hora (La hora en la cual se va a realizar el que hacer).
\item string Lugar (El lugar en el cual se realziara el que hacer).
\item string Titulo (el nombre de la tarea que se llevara a cabo).
\item string Descripcion (Explicar lo que se realizara en la actividad).

\end{itemize}

\section{"Que haceres"}
\begin{enumerate}
\item Agregar: cargar un nuevo elemento a la lista de que hacer.
\item Buscar: buscar un elemento dentro de una lista por medio de un identificador.
\item Borrar: borrar los elementos dentro de una lista.
\item Guardar: guardar el elemento a la lista de que hacer.
\item Actualizar: actualizar la fecha de entrada de los nuevos elementos.
\end{enumerate}

\begin{figure}[t!]

\includegraphics[width=2cm]{Universidad-del-Istmo-UNIS.png}
\end{figure}





\end{document}